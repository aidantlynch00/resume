%----------------------------------------------------------------------------------------
%	PACKAGES AND OTHER DOCUMENT CONFIGURATIONS
%----------------------------------------------------------------------------------------

\documentclass{resume} % Use the custom resume.cls style

\usepackage[left=0.40in,top=0.3in,right=0.75in,bottom=0.1in]{geometry} % Document margins
\usepackage{fontawesome}
\usepackage{times}
\usepackage[hidelinks]{hyperref}
\newcommand{\tab}[1]{\hspace{.2667\textwidth}\rlap{#1}}
\newcommand{\itab}[1]{\hspace{0em}\rlap{#1}}

\name{Aidan Lynch} % Your name 
\address{
\faPhone { (281) 665-9846 }
\faEnvelope { \href{ mailto:aidantlynch00@gmail.com } { aidantlynch00@gmail.com } }
\faLinkedin{ \href{ https://www.linkedin.com/in/aidan-lynch-935317194/ } { Aidan Lynch } }
\faGithub{ github.com/aidantlynch00} 
\faGlobe{ \href { https://www.aidantlynch.com } { aidantlynch.com } }
}

\begin{document}
{\centerline {\em \textbf { Seeking a software engineering position for Q1 2025. } } }
%----------------------------------------------------------------------------------------
%	EDUCATION SECTION
%----------------------------------------------------------------------------------------

\begin{rSection}{Education}

{\bf Rochester Institute of Technology} \textbar{ Rochester, NY} \hfill {\bf{\em August 2019 - December 2022}}
\newline
{ B.S. in Computer Science }
\hspace*{24pt}{ GPA: \bf{3.86} }

\begin{tabular}{ @{} >{\bfseries}l @{\hspace{0.4ex}} l }
CS Courses: \ & Computer Vision \textbullet{} Machine Learning \textbullet{} Algorithms \textbullet{} Networking \textbullet{} Distributed Systems
\\
Math Courses: \ & Graph Theory \textbullet{} Game Theory \textbullet{} Multivariable Calculus \textbullet{} Differential Equations
\end{tabular}

\end{rSection}

\begin{rSection}{Experience}
{\bf C Speed: } {\bf Software Engineer} \hfill {\bf{\em February 2023 - Present}}
%\\- Led the development of software for controlling UVC camera systems, automating previously manual routines.
\\- Translated project requirements into high-quality, maintainable software across multiple domains.
\\- Led multiple software development efforts while delegating responsibilities to interns and managing their deliverables.
\\- Improved and maintained build systems to create bootloader, kernel, and root file system images for various embedded devices.
\\- Leveraged knowledge in PowerShell and WSL to mount LUKS encrypted partitions on a Windows machine.
\\- {\bf Utilized: } C++, Bash, PowerShell, Linux/WSL, OpenEmbedded, OpenWRT, Azure DevOps, SVN

{\bf Bryx: } {\bf Software Engineering Intern} \hfill {\bf{\em May 2022 - August 2022}}
\\- Improved form extensibility through the use of a nested data structure for storing user-created forms.
\\- Built a dynamic rendering system for the form structure by using a set of recursive functions.
%\\- Implemented a drag-and-drop interface to create and arrange custom form elements.
\\- Incorporated regulatory form requirements by extending the nested structure for validation criteria.
\\- {\bf Utilized: } TypeScript, React, JSON, Material UI

{\bf C Speed: } {\bf Software Engineering Intern} \hfill {\bf{\em January 2021 - August 2021}}
\\- Improved a radar data parser's performance by 25\% through the identification of inefficient operations.
\\- Implemented an interactive website for the visualization of live and historical radar data.
%\\- Redesigned the radar data aggregation pipeline to run on a cloud machine for worldwide access.
% WHY DO I NEED TO DO THIS TO LEFT ALIGN ???
\\-\hspace*{2.5pt}Implemented a distributed algorithm for determining the temporal ranges of data points to improve search and playback.
\\- {\bf Utilized: } C++, C\#, JavaScript, AWS

\end{rSection}

%--------------------------------------------------------------------------------
%    Projects And Seminars
%-----------------------------------------------------------------------------------------------
\begin{rSection}{Projects}

%{\bf Compute Share: } {\em Rust, Git} \hfill {\em December 2022 - Present}
%\\- Designed system for requesting and performing distributed training for machine learning models. 
%\\- Researched distributed training techniques and its support in popular machine learning libraries.

%{\bf Puck Chasing: } \hfill {\bf{\em February 2025 - Present}}
%\\-
%\\-
%\\-

{\bf AutoPot: } \hfill {\bf{\em June 2023 - Present}}
\\- Designed a distributed system that separated sensor and control processes to improve flexibility and fault handling.
\\- Leveraged Unix domain sockets to handle inter-process communication to coordinate system behavior.
\\- Translated knowledge in build systems and shell scripting to improve the build, deployment, and testing cycle.
\\- {\bf Utilized: } Rust, Cross Compilation, SPI, SQLite

{\bf Automata: } \hfill {\bf{\em January 2025 - February 2025}}
\\- Implemented a generic, parallel compute engine to produce the next generation for several different cellular automata.
\\- Employed flamegraphs and profiling to highlight functions needing performance improvements.
\\- Reduced hot path execution time by 50\% through implementation of a partial rendering technique.
\\- {\bf Utilized: } Rust, Parallel Computing

%{\bf Monte Retires: } \hfill {\bf{\em December 2023 - March 2024}}
%\\- Implemented a Monte Carlo simulation using historical market data and custom investment strategies to calculate retirement valuations over time.
%\\- Integrated graphing functionality in order to visually compare the effectiveness of different strategies.
%\\- {\bf Utilized: } Python, Matplotlib, Monte Carlo

%{\bf Artificial Life: } \hfill {\bf{\em January 2022 - February 2022}}
%\\- Designed creatures with a genome to dictate traits and a neural network brain to govern behavior.
%\\- Implemented the NeuroEvolution of Augmenting Topologies algorithm to simulate evolution in a digital population.
%\\- Leveraged Perlin noise to guide the generation of random world terrain and obstacles.
%\\- {\bf Utilized: } C++, SFML, NEAT

%{\bf Personal Website: } \hfill {\bf{\em January 2024}}
%\\- Developed a website to showcase my personal projects and various networking resources.
%\\- Leveraged React to create reusable components for navigation and project cards.
%\\- {\bf Utilized: } React

%{\bf Piet Interpreter: } \hfill {\bf{\em March 2022 - April 2022}}
%\\- Developed a parser to interpret images representing Piet programs, translating pixels into executable instructions.
%\\- Implemented an interpreter to handle control flow and execution as well as manipulate program data.
%\\- {\bf Utilized: } Rust, Git

{\bf Drone: } \hfill {\bf{\em September 2018 - May 2019}}
\\- Leveraged the tiny footprint and compute power of a Raspberry Pi to control the flight of a custom built drone.
\\- Integrated user controls by converting radio receiver signals to thrust and translation vectors.
\\- Implemented a PID control loop to change propeller speed based on sensor readings and user input.
\\- {\bf Utilized: } C++, PID, GPIO, I$^2$C, Raspberry Pi

\end{rSection}

\begin{rSection}{Skills}

\begin{tabular}{ @{} >{\bfseries}l @{\hspace{0.4ex}} l }
Programming: \ & Rust \textbullet{} C \textbullet{} C++ \textbullet{} Bash \textbullet{} Python \textbullet{} Lua \textbullet{} C\# \textbullet{} Java \textbullet{} JavaScript \textbullet{} HTML \textbullet{} CSS  \textbullet{} SQL\\
Tools: \ & Git \textbullet{} SVN \textbullet{} Azure DevOps \textbullet{} Linux \textbullet{} OpenEmbedded \textbullet{} BuildRoot \textbullet{} Neovim
\end{tabular} 

\end{rSection}

%\begin{rSection}{Activities}

%{\bf Computer Science House, } {\em \textit{Member}} \hfill {\em August 2019 - December 2022}
%\\A special interest community at Rochester Institute of Technology based on academic and social excellence. With a strong focus on collaboration and %projects, Computer Science House helps members become better developers and better people.

%\end{rSection}

%\begin{center}
%\bf{References available upon request.}
%\end{center}

\end{document}